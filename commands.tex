%%%%%%%%%%%%%%%%%%%%%%%%%%%%%%%%%%%%%%%%%%%%%%%%%%%%%%%%%%%
%%%%%%%%%%%%%%%%%%%%%%%  Colors  %%%%%%%%%%%%%%%%%%%%%%%%%%
%%%%%%%%%%%%%%%%%%%%%%%%%%%%%%%%%%%%%%%%%%%%%%%%%%%%%%%%%%%

\definecolor{Blue}{RGB}{2,134,189}
\definecolor{Green}{RGB}{2, 188, 173}
\definecolor{Maroon}{RGB}{128, 0, 0}

%%%%%%%%%%%%%%%%%%%%%%%%%%%%%%%%%%%%%%%%%%%%%%%%%%%%%%%%%%%
%%%%%%%%%%%%%%%%%%%%%%%  Styles  %%%%%%%%%%%%%%%%%%%%%%%%%%
%%%%%%%%%%%%%%%%%%%%%%%%%%%%%%%%%%%%%%%%%%%%%%%%%%%%%%%%%%%

\newtcolorbox{classicboxstyle}[2][]{enhanced,colback=white,width={0.9\textwidth},
attach boxed title to top left={yshift={-0.5\baselineskip},xshift=0.5cm}, 
title={#2},
boxrule=0.5pt,
coltitle=black,
boxed title style={enhanced,
  borderline={0.1mm}{-0.1mm}{black,solid},
  colframe=white,
  colback=white,
  colupper={black},
},
#1%
}

\newtcolorbox{colorboxstyle}[3][]{enhanced,colback=white,width={0.9\textwidth},
attach boxed title to top left={yshift={-0.5\baselineskip},xshift=0.5cm}, 
title={#2},
boxrule=0.5pt,
coltitle=#3,
boxed title style={enhanced,
  borderline={0.1mm}{-0.1mm}{#3,solid},
  colframe=white,
  colback=white,
  colupper={black},
},
borderline={0.1mm}{0mm}{#3,solid}
#1%
}

%%%%%%%%%%%%%%%%%%%%%%%%%%%%%%%%%%%%%%%%%%%%%%%%%%%%%%%%%%%
%%%%%%%%%%%%%%%%%%%%%%%  Commands  %%%%%%%%%%%%%%%%%%%%%%%%
%%%%%%%%%%%%%%%%%%%%%%%%%%%%%%%%%%%%%%%%%%%%%%%%%%%%%%%%%%%


%boxes
\newcommand{\titlebox}[2]{
\begin{center}
\begin{classicboxstyle}{#1}
#2
\end{classicboxstyle}
\end{center}
}

\newcommand{\clrbox}[3]{
\begin{center}
\begin{colorboxstyle}{#1}{#3}
#2
\end{colorboxstyle}
\end{center}
}

%help commands
\newcommand\opt[1]{\textbf{#1}}
\newcommand\bopt[1]{\textcolor{blue}{\textbf{#1}}}
\newcommand\ropt[1]{\textcolor{red}{\textbf{#1}}}

%\newcommand\opr{\underline{\textbf{Определение}}. }
%\newcommand\teor[1]{\underline{\textbf{Теорема #1.}} }
\newcommand\ter{\underline{\textbf{Теорема.}} }

%proof
\newcommand\sproof{\textbf{Доказательство:} \\$\square$}
\newcommand\eproof{\text{\textbf{Что и требовалось доказать.}} \\$\blacksquare$}



\newcommand{\rnum}[1]{\uppercase\expandafter{\romannumeral  #1 \relax}}


\newcommand\opr[1]{\clrbox{Определение}{#1}{black}}
\newcommand\hlp[1]{\clrbox{Пояснения}{#1}{Blue}}
\newcommand\templ[2][]{\clrbox{Пример #1}{#2}{Green}}
\newcommand\teor[2][]{\clrbox{Теорема #1}{#2}{Maroon}}

%chapters without numerates
\newcommand\chap[1]{\chapter*{#1}\addcontentsline{toc}{chapter}{#1}}
\newcommand\sect[1]{\section*{#1}\addcontentsline{toc}{section}{#1}}
\newcommand\subsect[1]{\subsection*{#1}\addcontentsline{toc}{section}{#1}}

%enumarates
\newcommand\numi[1]{\begin{enumerate}[label=\arabic*)] #1 \end{enumerate}}
\newcommand\numr[1]{\begin{enumerate}[label=(\roman*)] #1 \end{enumerate}}
\newcommand\numc[1]{\begin{enumerate}[label=\alph*)] #1 \end{enumerate}}


%matrix
\newcommand\agg[2]{\left[ \begin{array}{#1} #2 \end{array} \right.}
\newcommand\sys[2]{\left\{ \begin{array}{#1} #2 \end{array} \right.}
\newcommand\opred[2]{\left| \begin{array}{#1} #2 \end{array} \right|}
\newcommand\chan[2]{\left| \begin{array}{#1} #2 \end{array} \right.}
\newcommand\many[2]{\begin{array}{#1} #2 \end{array} }

%img
\newcommand\imglink[1]{(рис. \ref{#1})}
\newcommand\addimg[1]{
  \begin{figure}[H]
        \centering
        \includegraphics[width=0.3\linewidth]{#1}
    \end{figure}
}

